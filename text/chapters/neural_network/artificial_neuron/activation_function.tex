An artificial neuron's activation function defines that neuron's output value
 for given inputs, commonly being f: R->R \cite{leskovec2020mining}. A significant trait of many activation functions is their differentiability, allowing them to be used for Backpropagation, ANN algorithm for training weights. Having derivative not saturating or exploding, heads towards 0 or inf, is necessary for activation functions.

For such reasons, the usage of step function or any linear function is unsuitable for ANN.

\setsecnumdepth{all}
\subsubsection{Sigmoid Function}
The sigmoid function is commonly used in ANN as an alternative to the step function. A popular choice of the sigmoid function is a logistic sigmoid. Its output value is in the range of 0 and 1.\\

MATH\\

One of the reasons being the simplicity of derivative calculation:\\

MATH\\

One of its disadvantages being the vanishing gradient. A problem where for a given very high or very low input values, there would be almost no change in its prediction. Possibly resulting in training complications or performance issues.\cite{7typesActivationFunction}
%=======================================================================================================================
\subsubsection{Hyperbolic Tangent}

Hyperbolic tangent is similar to logistic sigmoid function with a key difference in its output, ranging between -1 and 1. \\

MATH\\

It shares sigmoid's simple calculation of its derivative.\\

MATH\\

By being only moved and scaled version of the sigmoid function, hyperbolic tangent does share sigmoid's advantages and its disadvantages.\cite{leskovec2020mining}

%=======================================================================================================================
\subsubsection{Rectified Linear Unit}

The output of the rectified linear unit (ReLU) is defined as:\\

[MATH]\\

ReLU popularity is mainly for its computational efficiency.
[https://missinglink.ai/guides/neural-network-concepts/7-types-neural-network-activation-functions-right/]

ReLu's disadvantages appear when inputs approach zero or are negative. Causing the so-called dying ReLu problem, where the network is unable to learn.
There are many variations of ReLu to this date, e.g., Leaky ReLU, Parametric ReLU, ELU, ... 
%=======================================================================================================================
\subsubsection{Softmax}

Softmax separates itself from all the previously mentioned functions by its ability to handle multiple input values in the form of a vector x = (x1,x2,...,xn) and output for each xi defined as:\\

MATH\\

For output being normalized probability distribution, ensuring SUM MATH. [12] It is being used as the last activation function of ANN to normalize the network's output into n probability groups.

%=======================================================================================================================