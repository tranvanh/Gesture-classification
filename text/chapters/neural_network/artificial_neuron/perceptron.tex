Perceptron is the simplest class of artificial neurons developed by Frank Rosenblatt in 1958.[Rosenblatt, F. (1958). "The Perceptron: A Probabilistic Model For Information Storage And Organization In The Brain". Psychological Review. 65 (6): 386–408. CiteSeerX 10.1.1.588.3775. doi:10.1037/h0042519. PMID 13602029.]
Perceptron takes several binary inputs, vector x = (x1, x2,...,xn), and outputs a single binary number. \\
To express the importance of respected input edges, perceptron uses real numbers called weights,
 assigned to each edge, vector w = (w1,w2,...,wn).

A step function calculates the perceptron's output.
The function output is either 0 or 1 determined by whether its weighted sum w (SUM) is less or greater than its threshold value,
 a real number, usually represented as an incoming edge with a negative weight -1. \\
 \\
 FIGURE \\
 MATH \\
