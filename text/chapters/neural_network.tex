An artificial neural network (ANN) is a mathematical model mimicking biological neural networks,
namely their ability to learn and correct errors from previous experience.[Chen, Yung-Yao; Lin, Yu-Hsiu; Kung, Chia-Ching; Chung, Ming-Han;
Yen, I.-Hsuan (January 2019). "Design and Implementation of Cloud Analytics-Assisted Smart Power Meters Considering Advanced Artificial Intelligence as
 Edge Analytics in Demand-Side Management for Smart Homes". Sensors. 19 (9): 2047. doi:10.3390/s19092047. PMC 6539684. PMID 31052502.]
ANN consists of a collection of nodes called neurons, and each node is linked to other nodes via connections called synapses.
ANN is typically divided into an input layer, followed by several hidden layers and an output layer. [5M] \\
The ANN subject was first introduced by Warren McCulloch and Walter Pitts in "A logical calculus of the ideas immanent in nervous activity" published in 1943.[6 M] But it was not until recent years when ANN has gained popularity with still increasing advancements in technology and availability of training data.
ANN had become one of the default solutions for complex tasks which were previously thought be unsolvable by computers. [7M] \\
This chapter will briefly explore different types of neural units and their activation functions, along with some exemplary network architectures.

\setsecnumdepth{all}
\section{Artificial Neuron}
As previously mentioned, artificial neurons are units mimicking behaviors of biological neurons.
Meaning it can receive as well as pass information between themselves.

\setsecnumdepth{all}
\subsection{Perceptron}
\textbf{Perceptron} is the simplest class of artificial neurons developed by Frank Rosenblatt in 1958.\cite{perceptronprobabmodel}

Perceptron takes several binary inputs, vector $\vec{x} = (x_1, x_2,...,x_n)$, and outputs a single binary number. To express the importance of respected input edges, perceptron uses real numbers called weights, assigned to each edge, vector $\vec{w} = (w_1,w_2,...,w_n)$.

A \textit{step function} calculates the perceptron's output.
The function output is either 0 or 1 determined by whether its weighted sum $\alpha = \sum_{i} x_i w_i$ is less or greater than its \textit{threshold} value, a real number, usually represented as an incoming edge with a negative weight -1.\cite{matous}

\begin{equation}
    output =
\begin{cases}
    1, & \text{if $\alpha\ \geq\ threshold$}\\
    0, & \text{if $\alpha\ <\ threshold$}
\end{cases} 
\end{equation} 


\begin{figure}[h]
	\centering
    \includegraphics[width=12cm]{perceptron.png}
	\caption{Perceptron \cite{matous}}
	\label{fig:perceptron}
\end{figure}

%=======================================================================================================================
\subsection{Sigmoid Neuron}
\textbf{Sigmoid neuron}, similarly to perceptron, has inputs $\vec{x}$ and weights. The key difference comes in once we inspect the output value and its calculation. Instead of perceptron's binary output 0 or 1, a sigmoid neuron outputs a real number between 0 and 1 using a \textit{sigmoid function}.\cite{nndl2015michaelnielsen}\cite{rojas2013neural}\cite{matous}

\begin{equation}
    {\sigma(\alpha) = \frac{1}{1 + e^{-\alpha}}}
\end{equation}

\begin{figure}[h]
	\centering
    \subfloat[\centering Step function]{{\includegraphics[width=6cm]{step}}}%
    \qquad
    \subfloat[\centering Sigmoid function]{{\includegraphics[width=6cm]{sigmoid}}}%
    \caption{Comparison between step function and sigmoid function}
\end{figure}
As shown in Figure 1.1, the sigmoid function(1.1a) is a smoothed-out version of the step function(1.1b).

%=======================================================================================================================
\subsection{Activation Function}
An artificial neuron's activation function defines that neuron's output value for given inputs, commonly being f: R->R [11 M]. A significant trait of many activation functions is their differentiability, allowing them to be used for Backpropagation, ANN algorithm for training weights. Having derivative not saturating or exploding, heads towards 0 or inf, is necessary for activation functions.

For such reasons, the usage of step function or any linear function is unsuitable for ANN.

\setsecnumdepth{all}
\subsubsection{Sigmoid Function}
The sigmoid function is commonly used in ANN as an alternative to the step function. A popular choice of the sigmoid function is a logistic sigmoid. Its output value is in the range of 0 and 1.\\

MATH\\

One of the reasons being the simplicity of derivative calculation:\\

MATH\\

One of its disadvantages being the vanishing gradient. A problem where for a given very high or very low input values, there would be almost no change in its prediction. Possibly resulting in training complications or performance issues.
[https://missinglink.ai/guides/neural-network-concepts/7-types-neural-network-activation-functions-right/]
%=======================================================================================================================
\subsubsection{Hyperbolic Tangent}

Hyperbolic tangent is similar to logistic sigmoid function with a key difference in its output, ranging between -1 and 1. \\

MATH\\

It shares sigmoid's simple calculation of its derivative.\\

MATH\\

By being only moved and scaled version of the sigmoid function, hyperbolic tangent does share sigmoid's advantages and its disadvantages.\cite{leskovec2020mining}

%=======================================================================================================================
\subsubsection{Rectified Linear Unit}

The output of the rectified linear unit (ReLU) is defined as:\\

[MATH]\\

ReLU popularity is mainly for its computational efficiency.
[https://missinglink.ai/guides/neural-network-concepts/7-types-neural-network-activation-functions-right/]

ReLu's disadvantages appear when inputs approach zero or are negative. Causing the so-called dying ReLu problem, where the network is unable to learn.
There are many variations of ReLu to this date, e.g., Leaky ReLU, Parametric ReLU, ELU, ... 
%=======================================================================================================================
\subsubsection{Softmax}

Softmax separates itself from all the previously mentioned functions by its ability to handle multiple input values in the form of a vector x = (x1,x2,...,xn) and output for each xi defined as:\\

MATH\\

For output being normalized probability distribution, ensuring SUM MATH. [12] It is being used as the last activation function of ANN to normalize the network's output into n probability groups.

%=======================================================================================================================
%=======================================================================================================================
%=======================================================================================================================
\section{Types of Neural Networks}

To this day, there are many types and variations of ANN, each with its structure and use cases. Here we will briefly introduce the most common ones, such as feed-forward networks, convolutional neural networks, or recurrent neural networks.

\setsecnumdepth{all}
\subsection{Feed-forward Networks}
{\color{red}Feed-forward network (FNN) first ANN to be invetned and also the simplest form of ANN. It's name comes from how the infromation flows through the network. Its data travels in one direction, oriented from the input layer to the output layer, without cycles.\cite{ffnbrilliant} 
}
FNN may or may not contain several hidden layers of various widths. By having no back-loops, FNN generally minimizes error in its prediction by using the backpropagation algorithm to update its weight values.\cite{mainTypesANN}

GRAPH

The input layer takes input data, vector x, producing y at the output layer. The process of training weights
 consists of minimizing the loss function L(y,y), y being the target output of input x.\cite{lipton2015critical}

%=======================================================================================================================
\subsubsection{Backpropagation}
Backpropagation, short of backward propagation of errors, is a widely used algorithm in training FFN using gradient descent to update the weights. \cite{birlliantbackprop}

MATH

Its generalization is used for other ANNs. Providing a way to compute the gradient of the cost function,
 real number value expressing prediction incorrectness.\cite{Goodfellow-et-al-2016}

%=======================================================================================================================
\subsection{Convolutional Neural Networks}

Convolutional Neural network (CNN) is a variation of FNN primarily used for image classification,
 object recognition, and other two-dimensional inputs.\cite{Goodfellow-et-al-2016} CNN is usually consisting of the input layer followed by multiple hidden layers, typically several convolutional layers with standard pooling layers, and ending with the output layer.

\setsecnumdepth{all}
\subsubsection{Convolutional Layer}

The convolutional layers' objective is to extract key features from the input image by passing
 a matrix known as a kernel over the input image abstracted into a matrix.\cite{mathworkscnn}

IMAGE \\

The convolution result can be of two types. One being the convolved feature is reduced in dimensionality compared to the input, valid padding, and the other in which the dimensionality is either increased or remains the same, same padding. latter. [https://towardsdatascience.com/a-comprehensive-guide-to-convolutional-neural-networks-the-eli5-way-3bd2b1164a53]

\subsubsection{Pooling Layer}


Similar to the previously mentioned convolutional layer, the pooling layer reduces the convolved
 feature's spatial size to decrease the computational power required for data processing.
Furthermore, being useful by extracting dominant features, which are rotational and positional invariant,
 thus maintaining the process of effectively training the model.\cite{compguideCnn}

There are two types of pooling: max pooling and average pooling.
 Max Pooling returns the maximum value from the portion of the image covered by the kernel.
 It performs as a noise suppressant, discarding the noisy activations altogether and
  performing de-noising and dimensionality reduction. Where average pooling returns the
   average of all the values from the same covered portion, performing dimensionality reduction as a noise suppressing mechanism.
    Hence, it is possible to note that max-pooling performs better.\cite{compguideCnn}

IMAGE
%=======================================================================================================================
\subsection{Recurrent Neural Networks}
Recurrent Neural Network (RNN) is distinguished by its memory, taking input sequence with no predetermined size. Its past predictions influence currently generated output. Thus for the same input, RNN could produce different results depending on previous inputs in the sequence.\cite{rnnDSmedium}.

{\color{red}
RNNs features make it commonly used in fields such as speech recognition, image captioning, natural language processing or language translation. Some of populars being for example Siri, Google Translate or Google Voice search.\cite{ibmrnn}

As previously mentioned RNN takes into consideration informations from previous inputs. Let us look at an idiom "feeling under the weahther", where for it to make sense, words have to be in a specific order. RNN needs to account positions of each word and use its information to predict the next word in the sequence. Each timestep represents a single word. In our case third timestep represents "the". Its hidden state holds informations of previous inputs, "feeling" and "under".\cite{ibmrnn}
}

SCHEMATIC

Figure 1.7b shows the network for each time step, i.e., at time $t$, the input $\vec{x_t}$ goes into the network to produce output $\hat{y}_t$, the next time step of the input is $x_{t+1}$ with additional input from the previous time step from the hidden state $h_{t}$. This way, the neural network looks at the current input and has the context from the previous inputs.
With this structure, recurrent units hold the past values, referred to as memory. Making it possible to work with a context in data.
\cite{rnnin6}

The recurrent unit is calculated as follows:

\begin{equation}
    {h_t = f(W_{x}x_t + W_{h}h_{t-1}+\vec{b_h})}
\end{equation}

$f()$ being the activation function, $W_x,W_h$ are weight matrixes, $x_t$ is the input, and $\vec{b_h}$ is the vector of bias parameters. Unit at time step $t=0$ is initialized to $(0,0,...,0)$. The output $\hat{y_t}$ is then calculated as:

\begin{equation}
    {\hat{y}_t = g(W_{y}h_t + \vec{b_y})}
\end{equation}

$g()$ also being an activation function, usually being softmax to ensure the output is in the desired class range. $W_y$ is the weight matrix and $\vec{b_y}$ being a vector of biases determined during the learning process.

Training RNNs uses a modified version of the backpropagation algorithm called backpropagation through time (BPTT), working by unrolling the RNN \cite{Goodfellow-et-al-2016}, calculating the losses across time steps, then updating the weights with the backpropagation algorithm. More on RNN in \cite{lipton2015critical} by Liton et al.






%=======================================================================================================================
\subsection{Long Short-Term Memory}




\subsubsection{Bidirectional Long Short-Term Memory}
%=======================================================================================================================
%=======================================================================================================================