% arara: xelatex
% arara: xelatex
% arara: xelatex


% options:
% thesis=B bachelor's thesis
% thesis=M master's thesis
% czech thesis in Czech language
% english thesis in English language
% hidelinks remove colour boxes around hyperlinks

\documentclass[thesis=B,english]{FITthesis}[2020/10/23]

%\usepackage[utf8]{inputenc} % LaTeX source encoded as UTF-8
% \usepackage[latin2]{inputenc} % LaTeX source encoded as ISO-8859-2
% \usepackage[cp1250]{inputenc} % LaTeX source encoded as Windows-1250

% \usepackage{subfig} %subfigures
% \usepackage{amsmath} %advanced maths
% \usepackage{amssymb} %additional math symbols

\usepackage{dirtree} %directory tree visualisation
\usepackage{xcolor}
\usepackage{xspace}
\usepackage{amsmath}
\usepackage{amssymb}
\usepackage{graphicx}
\usepackage{subfig}
\usepackage{float}
\usepackage{nameref}
\usepackage{algorithm}
\usepackage{algpseudocode}
\usepackage{pdfpages}

\graphicspath{ {./assets/} }
% % list of acronyms
% \usepackage[acronym,nonumberlist,toc,numberedsection=autolabel]{glossaries}
% \iflanguage{czech}{\renewcommand*{\acronymname}{Seznam pou{\v z}it{\' y}ch zkratek}}{}
% \makeglossaries

% % % % % % % % % % % % % % % % % % % % % % % % % % % % % % 
% EDIT THIS
% % % % % % % % % % % % % % % % % % % % % % % % % % % % % % 

\department{Department of Theoretical Computer Science}
\title{Gesture detector with Leap Motion sensor}
\authorGN{Viet Anh} %author's given name/names
\authorFN{Tran} %author's surname
\author{Viet Anh Tran} %author's name without academic degrees
\authorWithDegrees{Viet Anh Tran} %author's name with academic degrees
\supervisor{Ing. Tomáš Nováček}

\acknowledgements{First, I would like to thank my supervisor Ing. Tomáš Nováček, for his active support and guidance on and off my studies. I also wish to thank my friends, namely Ája, Nikky, Daniel, David, Matěj, for making the world a bit more colorful and Bc. Matouš Kozák for his never-ending help during my time at the faculty. Finally, to thank my mother for putting up with me my entire life.}


\abstractEN{Exploring ways to control the virtual environment is a popular goal of many human-computer interaction researchers. One of the approaches is using Leap Motion optical sensors, developed specifically to track hand and finger movements. The bachelor thesis focuses on utilizing Leap Motion sensors in real-time gesture recognition using neural networks. We used two layered bidirectional LSTM architecture to train static gestures along with dynamic gestures. The neural network was benchmarked on a publicly available ASL dataset acquiring 89.07\% using 5-fold cross-validation on 200 epochs. The architecture was ultimately trained using our dataset of 3861 samples for real-time deployment. We demonstrated that the pre-trained model is sufficient to be integrated into other applications, and we also discussed the current state of the MultiLeap library, developed for hand detection using more than one Leap Motion senzor at once. We compared results of using multiple sensors with MultiLeap with results of using one sensor.
}

\abstractCS{Zkoumání způsobů pro ovládání virtuálního prostředí je populárním cílem mnoha výzkumných prací v odvětví interakce člověka s počítačem. Jeden ze způsobů je použití Leap Motion optického senzoru, vyvíjeného specificky pro rozpoznávání pohybu ruky a prstů. Tato bakalářská práce se zaměřuje na využití Leap Motion senzorů k rozpoznávání gest v reálném čase za pomocí neuronové sítě. Využili jsme architekturu dvouvrstvé obousměrné LSTM k natrénování statických i dynamických gest. Neuronová síť byla otestovaná na veřejně dostupném ASL datasetu s výsledkem 89.07\% za použití 5-fold cross validace s 200 iteracemi. Architektura byla ve finále natrénovaná využitím našeho vlastního datasetu s 3861 vzorky pro rozpoznávání v reálném čase. Demonstrovali jsme, že náš předtrénovaný model je vhodný pro použití v jiných aplikacích a také jsme diskutovali aktuální stav MultiLeap knihovny, vyvíjené pro detekci ruky pomocí více Leap Motion senzorů najednou. Porovnali jsme výsledky více senzorů použitím MultiLeap knihovny s výsledky naměřené jedním senzorem.}
\placeForDeclarationOfAuthenticity{Prague}
\keywordsCS{rozpoznávání gest, dvouvrstvé obousměrné LSTM, MultiLeap, strojové učení, rekurentní neuronová síť, rozpoznávání v reálném čase}
\keywordsEN{gesture recognition, two-layered bidirectional LSTM, MultiLeap, machine learning, recurrent neural network, real-time recognition}
\declarationOfAuthenticityOption{1} %select as appropriate, according to the desired license (integer 1-6)
% \website{http://site.example/thesis} %optional thesis URL


%========================================================================================================
\begin{document}
%========================================================================================================
	%========================================================================================================
	\setsecnumdepth{part}
	\chapter{Introduction}\label{ch:introduction}
	Mouse and keyboard are considered to be default provider for human-computer interaction nowadays. But with the maturity in technology, namely virtual and extended reality, the need for computers to understand body language of a human is more and more present. Actions such as rotation or grabbing and moving an object in three-dimensional space, are unnatural if we were to use computer mouse, where its movement is limited to two-dimensional space. Oppose to performing the desired action by hands in our three-dimensional environment.

One of the proposed solutions for the issue is gesture recognition. Where a general idea is for computers to have the ability of recognizing gestures and performing actions base on them. Therefore, several devices were developed to process an image and yield useful data for gesture recognition. Some of them being Microsoft Kinect, a device where the main intention was to interpret whole body movement. Making it lacking in required accurracy for a hand gesture recognition. 

Other option would be using Leap Motion Controller. Developed specificly for tracking hand movements and extracting its features, such as positions of fingers, hand rotation and others. Its accurracy in finger detection is up to 0.01 mm.

Unfortunatly Leap Motion Controller has no official library for gesture recognition. Limiting developers utilizing the controller for its key features.\newpage\cleardoublepage
	%========================================================================================================
	\setsecnumdepth{all}
	\chapter{Neural Networks}\label{ch:neural_network}
	An artificial neural network (ANN) is a mathematical model mimicking biological neural networks,
namely their ability to learn and correct errors from previous experience.[Chen, Yung-Yao; Lin, Yu-Hsiu; Kung, Chia-Ching; Chung, Ming-Han;
Yen, I.-Hsuan (January 2019). "Design and Implementation of Cloud Analytics-Assisted Smart Power Meters Considering Advanced Artificial Intelligence as
 Edge Analytics in Demand-Side Management for Smart Homes". Sensors. 19 (9): 2047. doi:10.3390/s19092047. PMC 6539684. PMID 31052502.]
ANN consists of a collection of nodes called neurons, and each node is linked to other nodes via connections called synapses.
ANN is typically divided into an input layer, followed by several hidden layers and an output layer. [5M] \\
The ANN subject was first introduced by Warren McCulloch and Walter Pitts in "A logical calculus of the ideas immanent in nervous activity" published in 1943.[6 M] But it was not until recent years when ANN has gained popularity with still increasing advancements in technology and availability of training data.
ANN had become one of the default solutions for complex tasks which were previously thought be unsolvable by computers. [7M] \\
This chapter will briefly explore different types of neural units and their activation functions, along with some exemplary network architectures.

\setsecnumdepth{all}
\section{Artificial Neuron}
As previously mentioned, artificial neurons are units mimicking behaviors of biological neurons.
Meaning it can receive as well as pass information between themselves.

\setsecnumdepth{all}
\subsection{Perceptron}
\textbf{Perceptron} is the simplest class of artificial neurons developed by Frank Rosenblatt in 1958.\cite{perceptronprobabmodel}

Perceptron takes several binary inputs, vector $\vec{x} = (x_1, x_2,...,x_n)$, and outputs a single binary number. To express the importance of respected input edges, perceptron uses real numbers called weights, assigned to each edge, vector $\vec{w} = (w_1,w_2,...,w_n)$.

A \textit{step function} calculates the perceptron's output.
The function output is either 0 or 1 determined by whether its weighted sum $\alpha = \sum_{i} x_i w_i$ is less or greater than its \textit{threshold} value, a real number, usually represented as an incoming edge with a negative weight -1.\cite{matous}

\begin{equation}
    output =
\begin{cases}
    1, & \text{if $\alpha\ \geq\ threshold$}\\
    0, & \text{if $\alpha\ <\ threshold$}
\end{cases} 
\end{equation} 


\begin{figure}[h]
	\centering
    \includegraphics[width=12cm]{perceptron.png}
	\caption{Perceptron \cite{matous}}
	\label{fig:perceptron}
\end{figure}

%=======================================================================================================================
\subsection{Sigmoid Neuron}
\textbf{Sigmoid neuron}, similarly to perceptron, has inputs $\vec{x}$ and weights. The key difference comes in once we inspect the output value and its calculation. Instead of perceptron's binary output 0 or 1, a sigmoid neuron outputs a real number between 0 and 1 using a \textit{sigmoid function}.\cite{nndl2015michaelnielsen}\cite{rojas2013neural}\cite{matous}

\begin{equation}
    {\sigma(\alpha) = \frac{1}{1 + e^{-\alpha}}}
\end{equation}

\begin{figure}[h]
	\centering
    \subfloat[\centering Step function]{{\includegraphics[width=6cm]{step}}}%
    \qquad
    \subfloat[\centering Sigmoid function]{{\includegraphics[width=6cm]{sigmoid}}}%
    \caption{Comparison between step function and sigmoid function}
\end{figure}
As shown in Figure 1.1, the sigmoid function(1.1a) is a smoothed-out version of the step function(1.1b).

%=======================================================================================================================
\subsection{Activation Function}
An artificial neuron's activation function defines that neuron's output value for given inputs, commonly being f: R->R [11 M]. A significant trait of many activation functions is their differentiability, allowing them to be used for Backpropagation, ANN algorithm for training weights. Having derivative not saturating or exploding, heads towards 0 or inf, is necessary for activation functions.

For such reasons, the usage of step function or any linear function is unsuitable for ANN.

\setsecnumdepth{all}
\subsubsection{Sigmoid Function}
\input{chapters/neural_network/artificial_neuron/activation_function/sigmoid_function.tex}
%=======================================================================================================================
\subsubsection{Hyperbolic Tangent}
\input{chapters/neural_network/artificial_neuron/activation_function/hyperbolic_tangent.tex}
%=======================================================================================================================
\subsubsection{Rectified Linear Unit}
\input{chapters/neural_network/artificial_neuron/activation_function/relu.tex}
%=======================================================================================================================
\subsubsection{Softmax}
\input{chapters/neural_network/artificial_neuron/activation_function/softmax.tex}
%=======================================================================================================================
%=======================================================================================================================
%=======================================================================================================================
\section{Types of Neural Networks}

To this day, there are many types and variations of ANN, each with its structure and use cases. Here we will briefly introduce the most common ones, such as feed-forward networks, convolutional neural networks, or recurrent neural networks.

\setsecnumdepth{all}
\subsection{Feed-forward Networks}
{\color{red}Feed-forward network (FNN) first ANN to be invetned and also the simplest form of ANN. It's name comes from how the infromation flows through the network. Its data travels in one direction, oriented from the input layer to the output layer, without cycles.\cite{ffnbrilliant} 
}
FNN may or may not contain several hidden layers of various widths. By having no back-loops, FNN generally minimizes error in its prediction by using the backpropagation algorithm to update its weight values.\cite{mainTypesANN}

GRAPH

The input layer takes input data, vector x, producing y at the output layer. The process of training weights
 consists of minimizing the loss function L(y,y), y being the target output of input x.\cite{lipton2015critical}

%=======================================================================================================================
\subsubsection{Backpropagation}
Backpropagation, short of backward propagation of errors, is a widely used algorithm in training FFN using gradient descent to update the weights. \cite{birlliantbackprop}

MATH

Its generalization is used for other ANNs. Providing a way to compute the gradient of the cost function,
 real number value expressing prediction incorrectness.\cite{Goodfellow-et-al-2016}

%=======================================================================================================================
\subsection{Convolutional Neural Networks}

Convolutional Neural network (CNN) is a variation of FNN primarily used for image classification,
 object recognition, and other two-dimensional inputs.\cite{Goodfellow-et-al-2016} CNN is usually consisting of the input layer followed by multiple hidden layers, typically several convolutional layers with standard pooling layers, and ending with the output layer.

\setsecnumdepth{all}
\subsubsection{Convolutional Layer}

The convolutional layers' objective is to extract key features from the input image by passing
 a matrix known as a kernel over the input image abstracted into a matrix.\cite{mathworkscnn}

IMAGE \\

The convolution result can be of two types. One being the convolved feature is reduced in dimensionality compared to the input, valid padding, and the other in which the dimensionality is either increased or remains the same, same padding. latter. [https://towardsdatascience.com/a-comprehensive-guide-to-convolutional-neural-networks-the-eli5-way-3bd2b1164a53]

\subsubsection{Pooling Layer}


Similar to the previously mentioned convolutional layer, the pooling layer reduces the convolved
 feature's spatial size to decrease the computational power required for data processing.
Furthermore, being useful by extracting dominant features, which are rotational and positional invariant,
 thus maintaining the process of effectively training the model.\cite{compguideCnn}

There are two types of pooling: max pooling and average pooling.
 Max Pooling returns the maximum value from the portion of the image covered by the kernel.
 It performs as a noise suppressant, discarding the noisy activations altogether and
  performing de-noising and dimensionality reduction. Where average pooling returns the
   average of all the values from the same covered portion, performing dimensionality reduction as a noise suppressing mechanism.
    Hence, it is possible to note that max-pooling performs better.\cite{compguideCnn}

IMAGE
%=======================================================================================================================
\subsection{Recurrent Neural Networks}
Recurrent Neural Network (RNN) is distinguished by its memory, taking input sequence with no predetermined size. Its past predictions influence currently generated output. Thus for the same input, RNN could produce different results depending on previous inputs in the sequence.\cite{rnnDSmedium}.

{\color{red}
RNNs features make it commonly used in fields such as speech recognition, image captioning, natural language processing or language translation. Some of populars being for example Siri, Google Translate or Google Voice search.\cite{ibmrnn}

As previously mentioned RNN takes into consideration informations from previous inputs. Let us look at an idiom "feeling under the weahther", where for it to make sense, words have to be in a specific order. RNN needs to account positions of each word and use its information to predict the next word in the sequence. Each timestep represents a single word. In our case third timestep represents "the". Its hidden state holds informations of previous inputs, "feeling" and "under".\cite{ibmrnn}
}

SCHEMATIC

Figure 1.7b shows the network for each time step, i.e., at time $t$, the input $\vec{x_t}$ goes into the network to produce output $\hat{y}_t$, the next time step of the input is $x_{t+1}$ with additional input from the previous time step from the hidden state $h_{t}$. This way, the neural network looks at the current input and has the context from the previous inputs.
With this structure, recurrent units hold the past values, referred to as memory. Making it possible to work with a context in data.
\cite{rnnin6}

The recurrent unit is calculated as follows:

\begin{equation}
    {h_t = f(W_{x}x_t + W_{h}h_{t-1}+\vec{b_h})}
\end{equation}

$f()$ being the activation function, $W_x,W_h$ are weight matrixes, $x_t$ is the input, and $\vec{b_h}$ is the vector of bias parameters. Unit at time step $t=0$ is initialized to $(0,0,...,0)$. The output $\hat{y_t}$ is then calculated as:

\begin{equation}
    {\hat{y}_t = g(W_{y}h_t + \vec{b_y})}
\end{equation}

$g()$ also being an activation function, usually being softmax to ensure the output is in the desired class range. $W_y$ is the weight matrix and $\vec{b_y}$ being a vector of biases determined during the learning process.

Training RNNs uses a modified version of the backpropagation algorithm called backpropagation through time (BPTT), working by unrolling the RNN \cite{Goodfellow-et-al-2016}, calculating the losses across time steps, then updating the weights with the backpropagation algorithm. More on RNN in \cite{lipton2015critical} by Liton et al.






%=======================================================================================================================
\subsection{Long Short-Term Memory}




\subsubsection{Bidirectional Long Short-Term Memory}
%=======================================================================================================================
%=======================================================================================================================\newpage\cleardoublepage
	%========================================================================================================
	\setsecnumdepth{all}
	\chapter{Gesture Recognition}\label{ch:gesture_recognition}
	
Gestures are classified into static gestures and dynamic gestures. Group of static gestures consits of fixed gestures which are not relative to time, where group of dynamic gestures are time varying.

Hand and body gesture recognition had followed a conventional scheme of extracting key features via one or multiple preprocessing sensors and applying machine learning techniques on them.\cite{avola}

The field of gesture recognition gave birth to several image processing devices yielding useful data. One of them being Microsoft Kinect, a device where the main intention was to interpret whole-body movement, making it lacking in required accuracy for hand gesture recognition. 

\section{Leap Motion Controller}
%=======================================================================================================================

Another option would be using a Leap Motion Controller (LMC), developed specifically to track hand movements and extract its features, such as positions of fingers, hand rotation, and others.

LMC consists of two monochromatic IR cameras and three IR LEDs (emitters). 

\begin{figure}[h]
	\centering
    \includegraphics[width=8cm]{lmc_schematic.png}
	\caption{Schematic View of Leap Motion Controller}
	\label{fig:lmcScheme}
\end{figure}



The LMC's current API, Leap Motion Service, yields positions of extracted hand features. All the positional data about the hand and its features are represented in the coordinate system relative to the LMC's center point, positioned at the middle IR LED.\cite{LMCanalysis} The x- and z-axes lie in the camera sensors plane, with the x-axis running along the camera baseline. The y-axis is vertical, with positive values increasing upwards (in contrast to the downward orientation of most computer graphics coordinate systems). The z-axis has positive values increasing toward the user.\cite{tomasMultileap}

\begin{figure}[h]
	\centering
    \includegraphics[width=8cm]{leap_axes.png}
	\caption{Leap Motion Controller Axes}
	\label{fig:lmcScheme}
\end{figure}

Unfortunately, Leap Motion Controller has no official library for gesture recognition, limiting developers from utilizing the controller for its key features. Orion, Leap Motion tracking software build for virtual reality, used to have a gesture detector with its 3.0 version, but the detector is absent with the release of more accurate version 4.0.

\section{Methods}
%=======================================================================================================================

Gestures classification should be taken into account when choosing appropriate methods due to their time varying properties. As previously mentioned, gestures are classified into static and dynamic groups.

\subsection{Static Gesture Recognition}

Common methods for static gesture recognition are Support Vector Machines(SVM), ANN or pattern techniques.\cite{savaris}

\subsection{Dynamic Gesture Recognition}

\subsection{LSTM}
Many of the proposed methods focuses either on static gesture recognition or dynamic gesture recognition, but very few of them are actually utilized for both types at the same time. 

\newpage\cleardoublepage
	%========================================================================================================
	\setsecnumdepth{all}
	\chapter{MultiLeap}\label{ch:multileap}
	
In 2018, developers from UltraLeap had released an experimental build for Leap Motion tracking software, which provided data from all connected LMCs at once. Despite having this feature, the provided tracking information for the same hand was different from each sensor due to different points of origin. This problem was solved by MultiLeap library created by Tomáš Nováček et al. in \cite{tomasMultileap}, which merges the information from all sensors and returns unified stream of data. 

\section{Alignment of the tracking data}

To align tracking data, we must first determine the position of LMCs in order to place them in the virtual World. This can be achieved by data sampling and computing sensor's positions and rotations in relation to other LMCs. \cite{tomasMultileap}

\subsection{Data sampling}

The MultiLeap library allows a user to sample data using a semi-automatic sampling process, which does not require the user to focus too much on the data acquisition itself. Each sample consists of 20 points from the hand – the points represent the center of each finger joint. 

The sampling is enabled manually, but data are sampled automatically per every Leap Motion frame, approximately 60 times per second. The general idea of automatically sampling is to calibrate sensors using data from already calibrated devices. First, one sensor is marked as calibrated. The first marked sensor is either the first connected sensor or one selected by a user. Uncalibrated sensors start acquiring samples if the presented hand is in their field of view and at the same time in the field of view of any calibrated sensor. The sample consists of uncalibrated sensor's original data and fused data from all calibrated, to which is the hand visible. Once the sensor collects enough samples, it begins to compute the optimal translation and rotation of the device. The sensor is then marked as calibrated. The process is repeated until all sensors are calibrated. \cite{tomasMultileap}


Hands will then align automatically, but it is up to the user, performing the calibration, to cover enough space of the tracking area. Most importantly, it is best to have diverse data for more accurate alignment \cite{tomasMultileap}.

Considering the tracking data, where the hand is completely still, it will not have the necessary diversity in its samples. The deviation between collected tracking data is too insignificant. If we were to move the hand across the tracking area, having it rotated in various ways in various positions, the deviation of rotations and positions will be more evident, and the calculation of the alignment more precise. \cite{tomasMultileap}

Another option for calibration is a fully manual setting, allowing a user to set the position and rotation of sensors n the Unity environment. Values need to be calculated accurately for the alignment to have any use. The main advantage of this approach is having the possibility of tracking different parts of the tracked space with the sensors, for example, LMCs being back to each other. \cite{tomasMultileap}

The combined approach is also possible. First, making a rough calibration manually and eventually improved by the semi-automatic.

\subsection{Kabsch algorithm}

Kabsch algorithm \cite{kabsch} also known as Procrustes superimposition, was used to determine the rotation of sensors by calculating optimal rotation matrix minimizing the root mean squared deviation between two paired sets of points. The set of LMC sensor, which was connected first, serves as a reference sensor. The second set of points consists of the tracking information from any other sensor. The algorithm is repeated for each connected sensor, excluding the reference sensor. \cite{tomasMultileap}

The goal of the Kabsch algorithm is to compute the optimal translation rotation of P onto Q, where P and Q are sets of pair points that minimize the distance between the two sets. Both P and Q are represented as $N \times 3$ matrix. Each row consists of coordinates of every point. \cite{tomasMultileap}

\begin{equation}
    \begin{pmatrix}
        x_1 & y_1 & z_1\\
        x_2 & y_2 & z_2\\
        \vdots & \vdots & \vdots\\
        x_N & y_N & z_N
    \end{pmatrix}
\end{equation}

Coordinates of the first point are in the first row, the second point in the second row, and the $N$th point in the $N$th row.

The algorithm has two main steps, computing the optimal translation, computation of the optimal matrix.

The optimal translation can be easily found by being the offset between the averages of two sets of points. As for optimal rotation, we must first calculate the mean center of the points by subtracting the coordinates of the respective centroid from the point coordinates. The centroid $C_P$ for $P$ is computed as follows:

\begin{equation}
    {C_P = {\frac{\sum_{i=1}^{N}P_i}{N}}}
\end{equation}

The mean-center calculation of all points in P:

\begin{equation}
    {P_i = P_i - C_P}
\end{equation}

Then, the $3\times3$ cross-variance matrix between the points must be calculated as follows in matrix notation:

\begin{equation}
    {H = P^T Q}
\end{equation}

At last, we will extract the rotation from the covariance matrix using polar decomposition. The extraction can be done in more iterations, resulting in more accurate rotation calculation but requiring higher computation time in return.

\section{Data fusion}

If multiple sensors detect the hand, the fusion algorithm is used. In most cases, not all sensors detect the hands properly. One of the yield information provided by MultiLeap library is a \textit{confidence}, a float value ranging from 0.3 to 1, which denotes the confidence level of the tracking data corresponding Leap Motion frame. The purpose of confidence level is to give more weight to tracking data from the sensor, which detects the hand better, making the tracking more accurate even if two out of three sensors would send inaccurate tracking data. The confidence level is of value 0.3 when the palm normal is in a 90\textdegree and 1 when in 0\textdegree or 180\textdegree angle to Y-axis. MultiLeap does not use the confidence of 0 because even with the occlusion of fingers and hand, the tracking data still carries some information about the hand. After few experiments, the value 0.3 was determined to be the most suitable confidence level for minimal tracking data when the palm normal is in 90\textdegree angle to the Y-axis of the sensor. The mentioned approach resulted in following equation for \textit{confidence} computation:

\begin{equation}
    {confidence = (0.283699 \cdot angle^2)-(0.891268 \cdot angle)+1}
\end{equation}

The function transfers the angle, in radians, between the palm normal and the sensor's normal to the corresponding confidence level. \cite{tomasMultileap}

The confidence level is used to give weight to data from the sensor which detects the hand better, making the tracking more precise despite faulty data coming from other sensors.\newpage\cleardoublepage
	%========================================================================================================
	\setsecnumdepth{all}
	\chapter{Implementation}\label{ch:implementation}
	As briefly mentioned in the Introduction chapter, our goal is to utilize Leap Motion controllers combined with the pre-trained ANN model. 

We picked Python to be our primary language for training the ANN model, with the proposed DLSTM architecture by Avola D., Bernardi M. et al. \cite{avola}, along with the web-based interactive development environment Jupyter Notebook. One of the main reasons to pick Python was its wide range of libraries and scientific packages supporting machine learning tasks. Most importantly, Keras, a high-level deep learning API integrated with TensorFlow, enabling the user to create and train model structures in very few steps.

\section{Dataset Description}

Training and testing have been performed on a combination of two gathered datasets, ASL Dataset \cite{avola}, and SHREC 2017 dataset created in conjunction with \cite{shrec}.

\subsection{SHREC 2017 Dataset}

The SHREC dataset contains sequences of 14 dynamic hand gestures (grab, tap, expand, pinch, rotation clockwise, rotation counterclockwise, swipe right, swipe left, swipe up, swipe down, swipe X, swipe +, swipe V, shake). Each gesture was performed between 1 and 10 times by 28 participants in two ways, using one finger and the whole hand. All participants were right-handed. The length of sample gestures varies between 20 to 170 frames, making some samples too short. We solved this by using the padding technique to an acceptable value of $T=100$ and discarding samples where more than 50 frames have to be padded \cite{shrec}.

\subsection{ASL Dataset}

ASL Dataset has been created by Avola D., Bernardi M. et al. \cite{avola} as the result of lack of public datasets holding necessary information about hand joints.
The dataset consists of 30 hand gestures, 18 static gestures (1, 2-V, 3, 4, 5, 6-W, 7, 8, 9, A, B, C, D, H, I, L, X, and Y), and 12 dynamic gestures (bathroom, blue, finish, green, hungry, milk,
past, pig, store, and where). Gestures were collected from 20 different people. 13 were used to form the training set, while the remaining 7 formed a test set. Each person performed 30 hand gestures twice, once for each hand, and each gesture is composed of fixed 200 frames as oppose to frame varying SHREC dataset \cite{avola}. Small modifications had to be made since we wanted to utilize both datasets at the same time. We stripped ASL Dataset of its dynamic gestures and split frames of static gesture sample in half, acquiring 2 samples. Static gestures are, in theory, one frame stretched out through time. Therefore such modification should not have a negative impact on our trained model.


\newpage\cleardoublepage
	%========================================================================================================
	\setsecnumdepth{all}
	\chapter{Experiments}\label{ch:experiments}
	As mentioned in \nameref{ch:introduction} chapter, we would want to evaluate the recognition performance based on a number of connected LMC sensors. We mainly focused on the overall placement of sensors and their field of view, having a single connected sensor as a reference field of view and exploring its expansion using different layouts with a different number of connected sensors. 

We used the demo application and trained model as described in section \ref{real_time_recognition}. The model was trained on our original dataset from section \ref{handicrafted_dataset}. 

\subsection{Testing Method}

For each gesture we performed 1000 classifications. We did not exclude classifications with corrupt sequences such as randomly floating hand without any hand presented or when the LMC sensor did not get a correct\newpage\cleardoublepage
	%========================================================================================================
	\setsecnumdepth{all}
	\chapter{Conclusion}\label{ch:conclusion}
	The goal of the thesis was to utilize LeapMotion sensors in relation to gesture recognition, creating a pre-trained model and using it to evaluate the capabilities of the MultiLeap library.

We explored publicly available ASL and SHREC datasets, discovered possible feature mislabeling in the ASL dataset, and discussed the dataset's suitability for our purposes. In relation to the discussion, we created a simple way of sampling our dataset with a simplified ability to detect moving sequences while sampling dynamic gestures. We created a dataset consisting of 7 static gestures (fist, 1-pointing, 2-peace sign, 3, 4, 5-open fist, pinch) and 2 dynamic gestures(swipe left, swipe right). Our dataset served the purpose but is not optimal for any benchmark evaluation. Furthermore, the dataset lacks complexity as well as the number of users used for sampling. We suggest expanding the dataset in future works with the engagement of more users, increasing the gesture set as well as its complexity.

ASL dataset, despite its mislabeling, was used for benchmarking and performance evaluation in the testing environment. Our dataset was then used for real-time deployment. Both datasets applied on 4-layered LSTM as well as 2-layered bidirectional LSTM. The 4-LSTM showed promising high results in the testing environment but did not have the desired behavior in real-time deployment due to the inability to learn dynamic gestures, while 2-layered bidirectional LSTM performed well on both fronts. 

We also explored the optimal number of layers and dropout rates for bidirectional LSTMs, resulting in having 2 layers in combination with the 0.6 dropout rate, which is an optimal compromise between accuracy and required training time. As a result, the 2-layered bidirectional LSTM achieved 89.07\% accuracy performing 5-fold cross-validation.

Using the pre-trained model, we created a demo application for debugging and experimental purposes in the form of a simple console application, which supports the connection of multiple Leap Motion sensors. Despite not having an optimal dataset, we achieved to create a responsive classifier for static gestures and dynamic gestures, suitable to be integrated into other applications. 

We have conducted several experiments to evaluate the model's performance in the real-time environment as well as evaluate the performance of the MultiLeap library by using multiple Leap Motion sensors. We explored several setups and the way they can affect Leap Motion detection. We have pointed out issues with the current MultiLeap library alongside its promising results in classifying hand gestures with challenging angles while using multiple sensors. We did not explore all possible setups there are, but it was enough to have a general idea of the current MultiLeap state. We will revisit our setups and explore more in future works with improved MultiLeap.

\newpage\cleardoublepage


\bibliographystyle{iso690}
\bibliography{mybibliographyfile}

\setsecnumdepth{all}
\appendix

\chapter{Acronyms}
% \printglossaries
\begin{description}
	\item[ANN] Artificial Neural Network
	\item[RNN] Recurrent Neural Network
	\item[BRNN] Bidirectional Recurrent Neural Network
	\item[CNN] Convolutional Neural Network
	\item[LSTM] Long Short-Term Memory
	\item[BLSTM] Bidirectional Long Short-Term Memory
	\item[DLSTM] Deep Long Short-Tzerm Memory
	\item[ReLU] Rectified Linear Unit
	\item[LMC] Leap Motion Controller
	\item[LED] Light Emitting Diode 
	\item[SVM] Support Vector Machine
	\item[MLP] Multilayer Perceptron
	\item[ASL] American Sign Language
	\item[SHREC] Shape Retrieval Contest  
	\item[API] Application Programming Interface
\end{description}


\chapter{Contents of enclosed CD}

%change appropriately

\begin{figure}
	\dirtree{%
		.1 README.md\DTcomment{the Markdown file with description}.
		.1 executables\DTcomment{the directory with executables}.
		.2 Dataset\DTcomment{the directory of the original dataset}.
		.2 TrainedModel\DTcomment{the directory of trained model}.
		.2 DataSampler.exe\DTcomment{data sampling application}.
		.2 model\_trainig.py\DTcomment{model training Python script}.
		.2 GestureApp.exe\DTcomment{gesture recognition demo application}.
		.1 src\DTcomment{the directory of source codes}.
		.1 text\DTcomment{the directory of \LaTeX{} source codes of the thesis}.
		.1 environment.yml\DTcomment{configuration file for conda environment}.
		.1 BP\_Viet\_Anh\_Tran\_2021.pdf\DTcomment{the thesis text in PDF format}.
	}
\end{figure}



\end{document}
